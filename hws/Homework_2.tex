\documentclass[11pt]{article}

\usepackage{latexsym}
\usepackage{amsmath}
\usepackage{amssymb}
\usepackage{graphicx}
\usepackage{theorem}
\usepackage{enumerate}
\usepackage{tikz}
\usepackage{circuitikz}
\usepackage{hyperref}
\usepackage{changepage} 

% \newtheorem{problem}{Problem}
\newcounter{problem}
\newenvironment{problem}[2][]{\refstepcounter{problem}\par\medskip
   \noindent \textbf{Problem \theproblem~(#2 points):}}{\vspace{1cm}}
\newcounter{subproblem}[problem]
\renewcommand{\thesubproblem}{\roman{subproblem}}
\newenvironment{subproblem}[2][]{\refstepcounter{subproblem}\par\medskip
\begin{adjustwidth}{\parindent}{}
   \textbf{\thesubproblem. (#2 points):}}{\end{adjustwidth}}
\begin{document}

\title{DS-GA 1018: Homework 2}
\author{Due Monday October 15\textsuperscript{th} at 5:00 pm}
\date{}

\maketitle

\begin{problem}{20} Consider a causal AR(2) process of the form:
\begin{align}
    X_t = \phi_2 X_{t-2} + W_t,
\end{align}
with $0 < \phi_2 < 1$ and $W_t \sim \mathcal{N}(0, \sigma_w^2)$.
\begin{subproblem}{6}
    Assume that we have observations $\{x_1,x_2\}$. Derive the mean and variance of a future observation $x_t$ with $t>2$. (\textit{Hint: you'll need your solutions from Problem 4 of Homework 1.})
\end{subproblem}
\begin{subproblem}{2}
    What is the mean for $t=3$ and $t=4$. Explain the intuition behind this result.
\end{subproblem}
\begin{subproblem}{2}
    What is the covariance for $t=3$ and $t=4$. Explain the intuition behind this result.
\end{subproblem}
\begin{subproblem}{2}
   What is the mean and variance of $x_t$ as $t \to \infty$? Explain the intuition behind this result.
\end{subproblem}
\begin{subproblem}{6}
    Assume that we have observations $x_1$. Derive the mean vector and covariance matrix of a future set of observations $\{x_t, x_{t+1}\}$ with $t>1$.
\end{subproblem}
\begin{subproblem}{2}
   What is the mean vector and covariance matrix of $\{x_t,x_{t+1}\}$ for $t=2$? Explain the intuition behind this result.
\end{subproblem}
\end{problem}

\newpage

\begin{problem}{10} Consider the generalization of ARCH(1) model given by:
    \begin{align}
    R_t &= \delta + Y_t \\
    Y_t &= \sigma_t W_t, \quad W_t \sim N(0, 1) \\
    \sigma_t^2 &= \alpha_0 + \alpha_1 Y_{t-1}^2,
    \end{align}
    where $\alpha_0 > 0$, $1 > \alpha_1 > 0$, and $\delta$ is a constant value.

    \begin{subproblem}{2} Derive the mean $\mu_{R_t}$.
    \end{subproblem}
    
    \begin{subproblem}{6} Derive the  covariance $\gamma_{R_t, R_{t+h}}$.
    \end{subproblem}
    
    \begin{subproblem}{2} Is $R_t$ a (weak) stationary process? Justify your answer quantitatively.
    \end{subproblem}

\end{problem}

\begin{problem}{10}\label{prob:sigma_lim}
    Consider the latent space model we presented in class defined by:
    \begin{align}
        \mathbf{z}_t &= \mathbf{A} \mathbf{z}_{t-1} + \mathbf{w}_t \\
        \mathbf{x}_t &= \mathbf{C} \mathbf{z}_{t} + \mathbf{v}_{t}
    \end{align}
    where the latent space $\mathbf{z}$ is has dimension $d$ and the data $\mathbf{x}$ has dimension $n$. Our noise is being drawn from $\mathbf{w}_t \sim \mathcal{N}(0,\mathbf{Q})$ and $\mathbf{v}_t \sim \mathcal{N}(0,\mathbf{R})$.
    \begin{subproblem}{4}
       Assume that, $n=d$ and that we have $C = \alpha \mathbb{I}$ and $R = \sigma_v^2 \mathbb{I}$. Write the mean $\mu_{t|t} = \mu_{\mathbf{z}_t|\mathbf{z}_{t-1},\mathbf{x}_t}$ and covariance $\Sigma_{t|t}$ in terms of the mean $\mu_{t|t-1} = \mu_{\mathbf{z}_t|\mathbf{z}_{t-1}}$ and covariance $\Sigma_{t|t-1}$. Simplify as much as possible.
    \end{subproblem}
    \begin{subproblem}{2}
       What happens in the limit $\sigma_v \to 0$? Demonstrate the answer quantitatively and explain the intuition behind this limit.
    \end{subproblem}
    \begin{subproblem}{2}
       What happens in the limit $\sigma_v \to \infty$? Demonstrate the answer quantitatively and explain the intuition behind this limit.
    \end{subproblem}
    \begin{subproblem}{2}
       What happens in the limit $\alpha \to 0$? Demonstrate the answer quantitatively and explain the intuition behind this limit.
    \end{subproblem}
\end{problem}

\begin{problem}{5}
    Consider a modified version of our latent space model that depends on a set of \textbf{observed} values $\mathbf{y}_t$ as follows:
    \begin{align}
        \mathbf{z}_t &= \mathbf{A} \mathbf{z}_{t-1} + \mathbf{B} \mathbf{y}_t + \mathbf{w}_t,
    \end{align}
    otherwise all the other components are identical to those described in Problem \ref{prob:sigma_lim}.
    \begin{subproblem}{5}
        Derive how this new process changes the filtering step of our Kalman filtering.
    \end{subproblem}
\end{problem}

\end{document}